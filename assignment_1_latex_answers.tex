\documentclass{article}
\usepackage[utf8]{inputenc}
\usepackage{amsmath}

\title{Networks and Random Processes Assignment 1}
\author{Charlie Pilgrim - 1864704}
\date{October 2018}

\begin{document}

\maketitle


\section{Question 1}


This section considers a Simple Random Walk on ${1,...,L}$ with probabilities $p \in [0,1]$ and $q = 1-p$ to jump right and left respectively. 

Different boundary conditions are considered.

\subsection{Case 1 - Periodic}

Periodic boundary conditions, so $p(0,L) = q$ and $p(L,0) = p$

The transition matrix is:

$P = \begin{bmatrix}
    0 & p & 0 & \dots  & 0 & q \\
    q & 0 & p & \dots  & 0 & 0\\
    0 & q & 0 & \dots  & 0 & 0\\
    \vdots & \vdots & \vdots & \ddots & \vdots \\
    0 & 0 & 0 & \dots & 0 & p \\
    p & 0 & 0 & \dots & q & 0
\end{bmatrix}$

The process is irreducible, i.e. every state can, eventually, reach every other state. And there is a finite state space, so it has 1 unique stationary distribution.

The states can be laid out on a circle, and are symmetrical, so the stationary distribution is where all states have equal probabilities. 

$\pi = (1/L, 1/L,..., 1/L)$

for all $p \in (0,1)$ 

The stationary distribution is reversible only for the case $p=q=\frac{1}{2}$.


Case 2 - Closed

Transition matrix:



$P = \begin{bmatrix}
    q & p & 0 & \dots  & 0 & 0 \\
    q & 0 & p & \dots  & 0 & 0\\
    0 & q & 0 & \dots  & 0 & 0\\
    \vdots & \vdots & \vdots & \ddots & \vdots \\
    0 & 0 & 0 & \dots & 0 & p \\
    0 & 0 & 0 & \dots & q & p
\end{bmatrix}$

If $p = 1$, there is an absorbing state at L, and the stationary dsitribution is $\pi = (0,0,...,0,1)$. This stationary distribution is reversible, all terms in the detailed balance equations are zero. This is not irreducible, as the walk can only move from $s_x$ to $s_{x+1}$. 

If $q = 1$, there is an absorbing state at 1, and the stationary dsitribution is $\pi = (1,0,...,0,0)$. This stationary distribution is reversible, all terms in the detailed balance equations are zero. This is not irreducible, as the walk can only move from $s_x$ to $s_{x-1}$. 

If $p = q =\frac{1}{2}$, the process is irreducible, as every state can, eventually, be reached from every other state. There is a finite state space and so there is onle 1 stationary distribution. The sum of all columns equal 1, so there is a constant left eigenvector, and so the stationary distribution is $\pi = (1/L, 1/L,..., 1/L)$. This stationary distribution is reversible.

If $p,q \neq [0,1/2,1]$, then the process is irreducible, and it has a finite state space so there is only 1 stationary distribution. 




Question -> does stationary dsitribution need to be normalised? - Yes.
Question -> if sum of all columns is 1, is constant vector a prob distribution? - Yes. Left eigenvector is constant.
Question -> How to work out general case for closed SRW - done in lecture, in workbook. Method 1 is through writing recursion realtions. Often easier to chekc for detailed balance first, and to write a simpler recursion relation based on that. 

For p not equal to q, solution will be reversible.


Hello


\end{document}
