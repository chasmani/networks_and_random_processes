\documentclass{article}
\usepackage[utf8]{inputenc}
\usepackage{amsmath}

\title{Networks and Random Processes Assignment 1}
\author{Charlie Pilgrim - 1864704}
\date{October 2018}

\begin{document}

\maketitle


\section{Question 1}


This section considers a Simple Random Walk on ${1,...,L}$ with probabilities $p \in [0,1]$ and $q = 1-p$ to jump right and left respectively. 

Different boundary conditions are considered.

\subsection{Part A}

\subsubsection{Case 1 - Periodic}

Periodic boundary conditions, so $p(0,L) = q$ and $p(L,0) = p$

The transition matrix is:

$P = \begin{bmatrix}
    0 & p & 0 & \dots  & 0 & q \\
    q & 0 & p & \dots  & 0 & 0\\
    0 & q & 0 & \dots  & 0 & 0\\
    \vdots & \vdots & \vdots & \ddots & \vdots \\
    0 & 0 & 0 & \dots & 0 & p \\
    p & 0 & 0 & \dots & q & 0
\end{bmatrix}$

The process is irreducible, i.e. every state can, eventually, reach every other state. And there is a finite state space, so it has 1 unique stationary distribution.

The states can be laid out on a circle, and are symmetrical, so the stationary distribution is where all states have equal probabilities. 

$\pi = (1/L, 1/L,..., 1/L)$

for all $p \in (0,1)$ 

The stationary distribution is reversible only for the case $p=q=\frac{1}{2}$.


\subsubsection{Case 2 - Closed}

Transition matrix:

$P = \begin{bmatrix}
    q & p & 0 & \dots  & 0 & 0 \\
    q & 0 & p & \dots  & 0 & 0\\
    0 & q & 0 & \dots  & 0 & 0\\
    \vdots & \vdots & \vdots & \ddots & \vdots \\
    0 & 0 & 0 & \dots & 0 & p \\
    0 & 0 & 0 & \dots & q & p
\end{bmatrix}$

If $p = 1$, there is an absorbing state at L, and the stationary dsitribution is $\pi = (0,0,...,0,1)$. This stationary distribution is reversible, all terms in the detailed balance equations are zero. This is not irreducible, as the walk can only move from $s_x$ to $s_{x+1}$. 

If $q = 1$, there is an absorbing state at 1, and the stationary dsitribution is $\pi = (1,0,...,0,0)$. This stationary distribution is reversible, all terms in the detailed balance equations are zero. This is not irreducible, as the walk can only move from $s_x$ to $s_{x-1}$. 

If $p = q =\frac{1}{2}$, the process is irreducible, as every state can, eventually, be reached from every other state. There is a finite state space and so there is onle 1 stationary distribution. The sum of all columns equal 1, so there is a constant left eigenvector, and so the stationary distribution is $\pi = (1/L, 1/L,..., 1/L)$. This stationary distribution is reversible.

If $p,q \neq [0,1/2,1]$, then the process is irreducible, and it has a finite state space so there is only 1 stationary distribution. Looing at the detailed balance equations, we can find a recurrence relation of the form:



$\pi_{x-1} p= \pi_x p \text{ over } x = \{2,3,..., L-1\}$


$ \pi_x = \pi_{x-1} \frac{p}{q} $

By induction, this suggests a solution like

$\pi_x = \pi_1 (\frac{p}{q})^{x-1}$

This is a reversible distribution, and so must also be stationary. As the process is ergodic, this is the unique stationary distrubution.

We can check this distribution is staionary for a closed simple random walk with 4 states

 $\begin{bmatrix}
    \pi_1 \\
    \pi_1\frac{p}{q} \\
    \pi_1(\frac{p}{q})^2 \\
    \pi_1(\frac{p}{q})^3 
\end{bmatrix}
\begin{bmatrix}
    q & p & 0 & 0 \\
    q & 0 & p & 0\\
    0 & q & 0 & p\\
    0 & 0 & q & p
\end{bmatrix} = 
\begin{bmatrix}
    \pi_1 \\
    \pi_1\frac{p}{q} \\
    \pi_1(\frac{p}{q})^2 \\
    \pi_1(\frac{p}{q})^3 
\end{bmatrix}$

We also need to normalise the distribution, so the stationary distribution will be

$\pi_x = \dfrac{\pi_1 (\frac{p}{q})^{x-1}}{\sum_i^L{\pi_i}}$

\subsection{Part B - Absorbing}

$P = \begin{bmatrix}
    1 & 0 & 0 & \dots  & 0 & 0 \\
    q & 0 & p & \dots  & 0 & 0\\
    0 & q & 0 & \dots  & 0 & 0\\
    \vdots & \vdots & \vdots & \ddots & \vdots \\
    0 & 0 & 0 & \dots & 0 & p \\
    0 & 0 & 0 & \dots & 0 & 1
\end{bmatrix}$

The process is not irreducible, as no other states can be reached from state 1 or state L. 

The (normalised) stationary distributions are

$\pi_1 = [1,0,0,...,0]$

$\pi_2 = [0,0,0,...,1]$

$\pi_3 = [a,0,0,..,0,1-a] \text{ where } a \in [0,1]$



These distributions are reversible, looking at the detailed balance conditions:

$\pi(x)p(x,y) = \pi(y)p(y,x)$ 

All terms for all equations are zero, therefore it is reversible.


\bigskip




The absorption probability in site L is 

$h_k^L = P(X_n = L \text{ for some } n \geq 0|X_0 = k)$


$h_k^L = P(X_n = L \text{ for some } n \geq 0|X_0 = k)$

$h_k^L = P(X_n = L | X_0 = k)$

Considering k+1, k-1, and by law of total probability:

$h_{k}^L = P(X_n = L | X_1 = k+1, X_0 = k) * p + P(X_n = L | X_1 = k-1, X_0 = k) * q $

Using the Markov property:

$h_{k}^L = P(X_n = L | X_1 = k+1) \times p + P(X_n = L | X_1 = k-1) \times q $

\bigskip

$h_{k}^L = h_{k+1}^L \times p + h_{k-1}^L \times q $

And the boundary conditions are:

$h_{1}^L = 0 \text{ and } h_{L}^L = 1$

\bigskip

If $p=q$ then this recursion relation becomes 

$h_{k}^L = \dfrac{h_{k+1}^L + h_{k-1}^L}{2} $

This is linear interpolation between the two surrounding states, so the absorption probability is linear in k. Consdiering the boundary conditions, the solution is:

$h_{k}^L = \frac{k-1}{L-1}$

\bigskip

If $p\neq q$, we can solve the recursion realtion by considering the ansatz:


$h_{k}^L = \lambda^k $

$\lambda = p \lambda^2 + q$

This has roots:

$\lambda_1 = 1 \text{ and } \lambda_2 = q/p$



The general solution is of the form:

$h_k^L = a\lambda_1 + b\lambda_2$ 

$h_k^L = a + b(\frac{q}{p})^k$ 


Looking at the boundary conditions:

$h_1^L = 0 = a + b(\frac{q}{p})$

$h_L^L = 1 = a + b(\frac{q}{p})^L$

Subtracting the first equation from the second equation, and solving for b:

$b = \dfrac{1}{(\frac{q}{p})^{L}-\frac{q}{p}})$

$a = \dfrac{-1}{(\frac{q}{p})^{L-1}-1}$


\subsection{Part C - simulations}

A simple random walk with L=10 and closed boundary conditions was simulated 500 times, with a $p=0.6$.




\end{document}
