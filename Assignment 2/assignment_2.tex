\documentclass{article}
\usepackage[utf8]{inputenc}
\usepackage{amsmath}
\usepackage{graphicx}
\usepackage{float}


\graphicspath {}


\title{Networks and Random Processes Assignment 2}
\author{Charlie Pilgrim - 1864704}
\date{October 2018}

\begin{document}

\maketitle


\section{Kingman's Coalescent}

\subsection{A}

$N_t$ is the number of particles at time t with $N_0=L$. The process $(N_t : t \geq 0)$ has the state space $\{1,...,L\}$

\subsubsection{Transition Rate of the process}

$$r(n,n-1) = {L\choose 2} \, , \, n \geq 2$$

QUESTION - WHAT ABOUT SAME STATE?
$r(n,n) = $

QUESTION - WHAT ABOUT OTHER STATES - HOW TO WRITE IT?
$r(n, y) = \, , \, y \neq n,n-1$

\subsubsection{Generator}

This is a jump process, so the generator is

$$(\mathcal{L}f)(x) = \int_{\Re} r(x,y)[f(y)-f(x)]dy$$

For this process

$$(\mathcal{L}f)(n) = r(n,n-1)(f(n-1)-f(n))$$

$$(\mathcal{L}f)(n) = {n\choose 2} (f(n-1)-f(n))$$

\subsubsection{Master Equation}

The master equation is

$$\frac{d}{dt} \pi_t(n) = \pi_t(n+1)r(n+1,n) - \pi_t(n)r(n,n-1)$$

$$\frac{d}{dt} \pi_t(n) = \pi_t(n+1){n+1\choose 2} - \pi_t(n) {n \choose 2}$$

QUESTION - IS THIS RIGHT?
QUESTION - IS THE NOTATION OKAY?
QUESTION - WHAT ABOUT EDGES? 


\subsubsection{Ergodicity}

The process is ergodic.

\subsubsection{Absorbing States}

The unique absorbing state is $N = 1$.

\subsubsection{Stationary Distributions}

Let a distribution $\pi = [N=1, N=2,... ,N=L]$

The unique stationary distribution is 

$$\pi_0 = [1,0,...,0]$$


\subsection{B - Mean Time to Asorption}

The rate of coalescence, ie moving to the next state, for each state is

$$\lambda_n = r(n,n-1) = {n\choose 2} = \frac{n(n-1)}{2}$$

The times in each state are expnentially dsitributed as

$$f_t(n) = {n \choose 2} e^{-{n \choose 2}t}$$

The expected time in each state, or the waiting time, is given by 

$$\beta_n = \dfrac{1}{\lambda_n} = \frac{2}{n(n-1)}$$

The expected time to absorption is the sum of the expected waiting times in each of the states

$$E(T) = \sum_{n=2}^L\frac{2}{n(n-1)}$$

Bringing the 2 outside of the summand and splitting up into partial fractions

$$E(T) = 2\sum_{n=2}^L\frac{1}{n-1} - \frac{1}{n}$$


$$E(T) = 2(\frac{1}{1}-\frac{1}{2}+\frac{1}{2}-\frac{1}{3} + \frac{1}{3} + ... - \frac{1}{L-1} + \frac{1}{L+1} - \frac{1}{L})$$

All but the first and last terms cancel giving

$$E(T) = 2(1 - \frac{1}{L})$$


\subsection{C - Rescaled process}

Rescale the process to $N_t/L$

$$(\mathcal{L}^Lf)(n/L) = \frac{1}{L} {n\choose2} (f(\frac{n-1}{L}) - f(\frac{n}{L}))$$

Taylor expand and let $x=\frac{n}{L}$:

$$(\mathcal{L}^Lf)(x) = \frac{1}{L}\frac{n(n-1)}{2} (f(x) - \frac{1}{L}f'(x) + \frac{1}{L^2}f''(x) + O(\frac{1}{L^3}) - f(x))$$

Cancel terms, substitute $n=Lx$ and rearrange

$$(\mathcal{L}^Lf)(x) = (\frac{x^2}{2} - \frac{x}{2L})(- f'(x) + \frac{1}{L}f''(x) + O(\frac{1}{L^2}))$$

$$\lim_{L \to \infty} (\mathcal{L}^Lf)(x) = \frac{-x^2}{2}f'(x)$$



QUESTION - STATE SPACE?
QUESTION - INITIAL CONDITION?


\subsubsection{Deterministic}

We have

$$\frac{d}{dt}E(f(x)) = E((\mathcal{L^L}f)(x))$$

$$\frac{d}{dt}E(f(x)) = E(\frac{-x^2}{2}\frac{dx}{dt}))$$

??? IS THAT RIGHT AT ALL?




\subsection{D - Simulations}

QUESTION - WHAT IS THE ADD THIS LINE BIT OF CODE IN EMMA'S WORKBOOK

\section{Ornstein-Uhlenbeck process}

\subsection{A}

\subsection{B}

\subsection{C}

\section{Moran Model and Wright-Fisher diffusion}

\subsection{A}

\subsection{B}

\subsection{C}

\subsection{D}

\subsection{E}

\subsection{F}



\end{document}





